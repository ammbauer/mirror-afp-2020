\chapter{Conclusion and Related Work}
\section{Related Work}
With \citet{barker:next:2009}, our UPF shares the observation
that a broad range of access control models can be reduced to a
surprisingly small number of primitives together with a set of
combinators or relations to build more complex policies. We also share
the vision that the semantics of access control models should be
formally defined.  In contrast to \cite{barker:next:2009}, UPF
uses higher-order constructs and, more importantly, is geared towards
machine support for (formally) transforming policies and supporting
model-based test case generation approaches.

\section{Conclusion Future Work}
We have presented a uniform framework for modelling security
policies. This might be regarded as merely an interesting academic
exercise in the art of abstraction, especially given the fact that
underlying core concepts are logically equivalent, but presented
remarkably different from---apparently simple---security textbook
formalisations.  However, we have successfully used the framework to
model fully the large and complex information governance policy of a
national health-care record system as described in the official
documents~\cite{brucker.ea:model-based:2011} as well as network
policies~\cite{brucker.ea:formal-fw-testing:2014}. Thus, we have shown
the framework being able to accommodate relatively conventional
RBAC~\cite{sandhu.ea:role-based:1996} mechanisms alongside less common
ones such as Legitimate Relationships. These security concepts are
modelled separately and combined into one global access control
mechanism.  Moreover, we have shown the practical relevance of our
model by using it in our test generation system
\testgen~\cite{brucker.ea:theorem-prover:2012}, translating informal
security requirements into formal test specifications to be processed
to test sequences for a distributed system consisting of applications
accessing a central record storage system.

Besides applying our framework to other access control models, we plan
to develop specific test case generation algorithms. Such
domain-specific algorithms allow, by exploiting knowledge about the
structure of access control models, respectively the UPF, for a
deeper exploration of the test space. Finally, this results in an
improved test coverage.
