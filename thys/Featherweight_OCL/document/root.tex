\documentclass[fontsize=11pt,paper=a4,open=right,twoside,abstract=true]{scrreprt}
\usepackage{fixltx2e}
\usepackage{isabelle,isabellesym}
\usepackage[nocolortable, noaclist]{hol-ocl-isar}
\usepackage{booktabs}
\usepackage{graphicx}
\usepackage{amssymb}
\usepackage[numbers, sort&compress, sectionbib]{natbib}
\usepackage[caption=false]{subfig}
\usepackage{lstisar}
\usepackage{tabu}
\usepackage[]{mathtools}
\usepackage{prooftree}
\usepackage[english]{babel}
\usepackage[pdfpagelabels, pageanchor=false, plainpages=false]{hyperref}
% \usepackage[draft]{fixme}

% MathOCl expressions
\colorlet{MathOclColor}{Black}
\colorlet{HolOclColor}{Black}
\colorlet{OclColor}{Black}
%
\sloppy 

\uchyph=0
\graphicspath{{data/},{figures/}}
\allowdisplaybreaks

\renewcommand{\HolTrue}{\mathrm{true}}
\renewcommand{\HolFalse}{\mathrm{false}}
\newcommand{\ptmi}[1]{\using{\mi{#1}}}
\newcommand{\Lemma}[1]{{\color{BrickRed}%
    \mathbf{\operatorname{lemma}}}~\text{#1:}\quad}
\newcommand{\done}{{\color{OliveGreen}\operatorname{done}}}
\newcommand{\apply}[1]{{\holoclthykeywordstyle%
    \operatorname{apply}}(\text{#1})}
\newcommand{\fun} {{\holoclthykeywordstyle\operatorname{fun}}}
\newcommand{\definitionS} {{\holoclthykeywordstyle\operatorname{definition}}}
\newcommand{\where} {{\holoclthykeywordstyle\operatorname{where}}}
\newcommand{\datatype} {{\holoclthykeywordstyle\operatorname{datatype}}}
\newcommand{\types} {{\holoclthykeywordstyle\operatorname{types}}}
\newcommand{\pglabel}[1]{\text{#1}}
\renewcommand{\isasymOclUndefined}{\ensuremath{\mathtt{invalid}}}
\newcommand{\isasymOclNull}{\ensuremath{\mathtt{null}}}
\newcommand{\isasymOclInvalid}{\isasymOclUndefined}
\DeclareMathOperator{\inv}{inv}
\newcommand{\Null}[1]{{\ensuremath{\mathtt{null}_\text{{#1}}}}}
\newcommand{\testgen}{HOL-TestGen\xspace}
\newcommand{\HolOption}{\mathrm{option}}
\newcommand{\ran}{\mathrm{ran}}
\newcommand{\dom}{\mathrm{dom}}
\newcommand{\typedef}{\mathrm{typedef}}
\newcommand{\mi}[1]{\,\text{#1}}
\newcommand{\state}[1]{\ifthenelse{\equal{}{#1}}%
  {\operatorname{state}}%
  {\operatorname{\mathit{state}}(#1)}%
}
\newcommand{\mocl}[1]{\text{\inlineocl|#1|}}
\DeclareMathOperator{\TCnull}{null}
\DeclareMathOperator{\HolNull}{null}
\DeclareMathOperator{\HolBot}{bot}


% urls in roman style, theory text in math-similar italics
\urlstyle{rm}
\isabellestyle{it}
\newcommand{\ie}{i.\,e.\xspace}
\newcommand{\eg}{e.\,g.\xspace}
\renewcommand{\isamarkupheader}[1]{\chapter{#1}}
\renewcommand{\isamarkupsection}[1]{\section{#1}}
\renewcommand{\isamarkupsubsection}[1]{\subsection{#1}}
\renewcommand{\isamarkupsubsubsection}[1]{\subsubsection{#1}}

\begin{document}
\renewcommand{\subsubsectionautorefname}{Section}
\renewcommand{\subsectionautorefname}{Section}
\renewcommand{\sectionautorefname}{Section}
\renewcommand{\chapterautorefname}{Chapter}
\newcommand{\subtableautorefname}{\tableautorefname}
\newcommand{\subfigureautorefname}{\figureautorefname}

\title{Featherweight OCL}
\subtitle{A Proposal for a Machine-Checked Formal Semantics for OCL 2.5}
\author{%
  \href{http://www.brucker.ch/}{Achim D. Brucker}\footnotemark[1]
  \and
  \href{https://www.lri.fr/~tuong/}{Fr\'ed\'eric Tuong}\footnotemark[3]
  \and
  \href{https://www.lri.fr/~wolff/}{Burkhart Wolff}\footnotemark[2]}
\publishers{%
  \footnotemark[1]~SAP AG, Vincenz-Priessnitz-Str. 1, 76131 Karlsruhe,
  Germany \texorpdfstring{\\}{} \href{mailto:"Achim D. Brucker"
    <achim.brucker@sap.com>}{achim.brucker@sap.com}\\[2em]
  %
  \footnotemark[3]~Univ. Paris-Sud, IRT SystemX, 8 av.~de la Vauve, \\
  91120 Palaiseau, France\\
  frederic.tuong@\{u-psud, irt-systemx\}.fr\\[2em]
  %
  \footnotemark[2]~Univ. Paris-Sud, Laboratoire LRI, UMR8623, 91405 Orsay, France\\
  CNRS, 91405 Orsay, France\texorpdfstring{\\}{}
  \href{mailto:"Burkhart Wolff" <burkhart.wolff@lri.fr>}{burkhart.wolff@lri.fr}
}


\maketitle

\begin{abstract}
  The Unified Modeling Language (UML) is one of the few modeling
  languages that is widely used in industry. While UML is mostly known
  as diagrammatic modeling language (\eg, visualizing class models),
  it is complemented by a textual language, called Object Constraint
  Language (OCL). OCL is a textual annotation language, based on a
  three-valued logic, that turns UML into a formal language.
  Unfortunately the semantics of this specification language, captured
  in the ``Annex A'' of the OCL standard, leads to different
  interpretations of corner cases.  Many of these corner cases had
  been subject to formal analysis since more than ten years.

  The situation complicated when with version 2.3 the OCL was aligned
  with the latest version of UML: this led to the extension of the
  three-valued logic by a second exception element, called
  \inlineocl{null}.  While the first exception element
  \inlineocl{invalid} has a strict semantics, \inlineocl{null} has a
  non strict semantic interpretation. These semantic difficulties lead
  to remarkable confusion for implementors of OCL compilers and
  interpreters.

  In this paper, we provide a formalization of the core of OCL in
  HOL\@. It provides denotational definitions, a logical calculus and
  operational rules that allow for the execution of OCL expressions by
  a mixture of term rewriting and code compilation.  Our formalization
  reveals several inconsistencies and contradictions in the current
  version of the OCL standard.  They reflect a challenge to define and
  implement OCL tools in a uniform manner.  Overall, this document is
  intended to provide the basis for a machine-checked text ``Annex A''
  of the OCL standard targeting at tool implementors.

\end{abstract}

\tableofcontents
\section{Introduction}

Linearizability \cite{HerlihyWing90Linearizability} is a  key design  methodology  for reasoning
about implementations of concurrent abstract data types in both shared
memory  and message passing  systems.  It  presents the  illusion that
operations execute sequentially and fault-free, despite the asynchrony
and faults that are often present in a concurrent system, especially a
distributed one.

However,  devising complete  linearizable objects  is  very difficult,
especially  in  the presence  of  process  crashes and asynchrony, requiring  complex
algorithms   (such  as   Paxos \cite{Lamport98PartTimeParliament})  to   work  correctly   under  general
circumstances,  and  often  resulting  in bad  average-case  behavior.
Concurrent algorithm designers therefore resort to speculation, i.e. to optimizing existing
algorithms to handle common scenarios more efficiently.
More precisely, a speculative systems has a fall-back mode that works in all situations and several optimization modes, each of which is very efficient in a particular situation but might not work at all in some other situation. By observing its execution, a speculative system speculates about which particular situation it will be subject to and chooses the most efficient mode for that situation. If speculation reveals wrong, a new speculation is made in light of newly available observations.
Unfortunately, building speculative system ad-hoc results in protocols so complex that it is no longer
tractable to prove their correctness.

We present  an I/O-automaton \cite{Lynch89anintroduction} specification, called ALM (a shorthand for Abortable Linearizable Module), which can be
used to build a speculative linearizable algorithm out of independent modules that implement the different modes of the speculative algorithm. The ALM is at the heart of the Speculative Linearizability framework \cite{GKL2012SpeculativeLinearizability}.

The ALM automaton produces traces that are linearizable with respect to
a generic type of object.  Moreover,  the composition of two instances of
the  ALM  automaton behaves  like  a  single  instance.  Hence  it  is
guaranteed that the composition of  any number of instances of the ALM
automaton is linearizable.

The properties  stated  above greatly simplify the development and analysis of speculative systems: 
Instead of having to reason about an entanglement of complex protocols, one can devise several modules
with the property  that, when taken in isolation, each module refines
the  ALM automaton.  
Hence complex  protocols can be  divided into  smaller modules
that  can be  analyzed independently of each other.  In  particular, it allows  to
optimize  an existing  protocol  by creating  separate  optimization
modules, prove each optimization correct in isolation, and
obtain the correctness of the overall protocol from the correctness of the existing one.

In this document we define the ALM automaton and prove the Composition
Theorem, which states that the composition of two instances of the ALM
automaton behaves as a single instance  of the ALM automaton. We use a
refinement mapping to establish this fact.


\part{A Proposal for Formal Semantics of OCL 2.5}


\input{OCL_core.tex}
\input{OCL_lib.tex}
\input{OCL_state.tex}
\input{OCL_tools.tex}
\input{OCL_main.tex}


\part{Examples}
\chapter{The Employee Analysis Model}
\label{ex:employee-analysis}
\input{Employee_AnalysisModel_UMLPart.tex}
\input{Employee_AnalysisModel_OCLPart.tex}
\chapter{The Employee Design Model}
\label{ex:employee-design}
\input{Employee_DesignModel_UMLPart.tex}
\input{Employee_DesignModel_OCLPart.tex}


%%% Local Variables: 
%%% mode: latex
%%% TeX-master: "root"
%%% End:

\section{Conclusion}\label{sec:conclusion}

This work presented the Isabelle Collections Framework, an efficient and extensible collections framework for Isabelle/HOL.
The framework features data-refinement techniques to refine algorithms to use concrete collection datastructures,
and is compatible with the Isabelle/HOL code generator, such that efficient code can be generated for all supported target languages.
Finally, we defined a data refinement framework for the while-combinator, and used it to specify a state-space exploration algorithm
and stepwise refined the specification to an executable DFS-algorithm using a hashset to store the set of already known states.

Up to now, interfaces for sets and maps are specified and implemented using lists, red-black-trees, and hashing. Moreover, an amortized constant time 
fifo-queue (based on two stacks) has been implemented. However, the framwork is extensible, i.e. new interfaces, algorithms and implementations can easily be added and integrated with the existing ones.

\subsection{Future Work}
There are several starting points for future work:

\begin{itemize}
  \item Currently, the generic algorithms are instantiated semi-automatically by an ad-hoc ruby script and a
  manual description of the generic algorithms to be instantiated. However, the process of instantiating
  generic algorithms could be fully mechanized, if one would add support for specifying interfaces, 
  implementations and generic algorithms in Isabelle/HOL. 
  
  \item Generic algorithms are declared as functions that take the required functions as arguments.
  While this is convenient for small generic algorithms, it can become tedious for larger algorithms,
  involving many interfaces. 
  Luckily, the generic algorithms defined in this library are rather small. For bigger developments, we currently recommend to
  fix the used implementations, i.e. to define specific algorithms rather than generic ones. This is, for example, done in the 
  DFS-search example (Section~\ref{thy:Exploration_Example}), where we fix hashsets instead of defining a generic algorithm for
  all set implementations. Hence there is a need for exploring more convenient ways to specify generic algorithms.
  
  \item Currently, explicit invariants are used all over the library. While it would be more convenient to use implicit invariants (encoded in the type), this
  is (as of Isabelle2009) not supported by the code generator. As code-generation is a main-requirement of this library, we had to revert to explicit invariants.

  \item What is called hashsets in this library are not true hashsets, as mapping hashcodes to buckets is done by a red-black tree instead
      of an array. This is more convenient for proofs in Isabelle/HOL, and has the advantage that no rehashing is needed if the map grows bigger, but it is also slower than array-based hashmaps. There is support for true arrays in Isabelle by means of a Heap-Monad \cite{BKHEM08} with the support for efficient code generation, that could probably be used to implement hashmaps with arrays.
\end{itemize}

\subsection {Trusted Code Base}
  In this section we shortly characterize on what our formal proof depends, i.e. how to interpret the information contained in this formal proof and the fact that it
  is accepted by the Isabelle/HOL system.

  First of all, you have to trust the theorem prover and its axiomatization of HOL, the ML-platform, the operating system software and the hardware it runs on.
  All these components are, in theory, able to cause false theorems to be proven. However, the probability of a false theorem to get proven due to a hardware error 
  or an error in the operating system software is reasonably low. There are errors in hardware and operating systems, but they will usually cause the system to crash 
  or exhibit other unexpected behaviour, instead of causing Isabelle to quitely accept a false theorem and behave normal otherwise. The theorem prover itself is a bit more critical in this aspect. However, Isabelle/HOL is implemented in LCF-style, i.e. all the proofs are eventually checked by a small kernel of trusted code, containing rather simple operations. HOL is the logic that is most frequently used with Isabelle, and it is unlikely that it's axiomatization in Isabelle is inconsistent and no one found and reported this inconsistency already.

  The next crucial point is the code generator of Isabelle. We derive executable code from our specifications. The code generator contains another (thin) layer of untrusted code. This layer has some known deficiencies\footnote{For example, the Haskell code generator may generate variables starting with upper-case letters, while the Haskell-specification requires variables to start with lowercase letters. Moreover, the ML code generator does not know the ML value restriction, and may generate code that violates this restriction.} (as of Isabelle2009) in the sense that invalid code is generated. This code is then rejected by the target language's compiler or interpreter, but does not silently compute the wrong thing. 

  Moreover, assuming correctness of the code generator, the generated code is only guaranteed to be {\em partially} correct\footnote{A simple example is the always-diverging function ${\sf f_{div}}::{\sf bool} = {\sf while}~(\lambda x.~{\sf True})~{\sf id}~{\sf True}$ that is definable in HOL. The lemma $\forall x.~ x = {\sf if}~{\sf f_{div}}~{\sf then}~x~{\sf else}~x$ is provable in Isabelle and rewriting based on it could, theoretically, be inserted before the code generation process, resulting in code that always diverges}, i.e. there are no formal termination guarantees.

\subsection{Acknowledgement}
We thank Tobias Nipkow for encouraging us to make the collections framework an independent development. Moreover, we thank Markus M\"uller-Olm for discussion about data-refinement. Finally, we thank the people on the Isabelle mailing list for quick and useful response to any Isabelle-related questions.

\bibliographystyle{abbrvnat}
\bibliography{root} 

\end{document}

%%% Local Variables:
%%% mode: latex
%%% TeX-master: t
%%% End:

%  LocalWords:  implementors denotational OCL UML
