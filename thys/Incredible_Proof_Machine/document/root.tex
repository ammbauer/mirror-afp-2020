\documentclass[11pt,DIV16,a4paper,parskip=half]{scrartcl}
\usepackage{isabelle,isabellesym}
\usepackage[only,bigsqcap]{stmaryrd}

% From src/HOL/HOLCF/document/root
\newcommand{\isasymnotsqsubseteq}{\isamath{\not\sqsubseteq}}

\usepackage{amsmath}
\usepackage{amsfonts}
\usepackage{mathtools}
\usepackage[T1]{fontenc}
\usepackage[utf8]{inputenc}
\usepackage{calc}

\usepackage{floatpag}
\floatpagestyle{empty}

% this should be the last package used
\usepackage{pdfsetup}

% silence the KOMA script warnings
\def\bf{\normalfont\bfseries}
\def\it{\normalfont\itshape}
\def\rm{\normalfont\rmfamily}


% urls in roman style, theorys in math-similar italics
\urlstyle{rm}
\isabellestyle{it}

% Isabelle does not like *} in a text {* ... *} block
% Concrete implemenation thanks to http://www.mrunix.de/forums/showpost.php?p=235085&postcount=5
\newenvironment{alignstar}{\csname align*\endcsname}{\csname endalign*\endcsname}
\newenvironment{alignatstar}{\csname alignat*\endcsname}{\csname endalignat*\endcsname}


% Entering \ in Isabelle/jEdit has unwanted consequences
\catcode`\|=0 %

% Unfortunately, _ are the norm for Isabelle file names
\catcode`\_=12

\begin{document}

\title{The meta theory of the\\ Incredible Proof Machine}
\author{Joachim Breitner \and Denis Lohner}
\maketitle

\begin{abstract}
\noindent
The Incredible Proof Machine is an interactive visual theorem prover which represents proofs as
port graphs. We model this proof representation in Isabelle, and prove that it is just
as powerful as natural deduction.
\end{abstract}

\tableofcontents

\section{Introduction}

The Incredible Proof Machine (\url{http://incredible.pm}) is an educational tool that allows the user to
prove theorems just by dragging proof blocks (corresponding to proof rules) onto a canvas, and connecting them
correctly.

In the ITP 2016 paper \cite{incredible} the first author formally describes the
shape of these graphs, as port graphs, and gives the necessary conditions for
when we consider such a graph a valid proof graph. The present Isabelle
formalization implements these definitions in Isabelle, and furthermore proves
that such proof graphs are just as powerful as natural deduction.

All this happens with regard to an abstract set of formulas (theory
\isa{Abstract_Formula}) and an abstract set of logic rules (theory
\isa{Abstract_Rules}) and can thus be instantiated with various logics.

This formalization covers the following aspects:
\begin{itemize}
\item We formalize the definition of port graphs, proof graphs and the conditions for such a proof
      graph to be a valid graph (theory \isa{Incredible_Deduction}).
\item We provide a formal description of natural deduction (theory \isa{Natural_Deduction}), 
      which connects to the existing theories in the AFP entry “Abstract
      Completeness” \cite{Abstract_Completeness-AFP}.
\item For every proof graph, we construct a corresponding natural deduction derivation tree
      (theory  \isa{Incredible_Correctness}).
\item Conversely, if we have a natural deduction derivation tree, we can construct a proof graph
      thereof (theory \isa{Incredible_Completeness}).
      
      This is the much harder direction, mostly because the freshness side condition for locally
      fixed constants (such as in the introduction rule for the universal quantifier) is a local
      check in natural deduction, but a global check in proofs graphs, and thus some elaborate
      renaming has to occur (\isa{globalize} in \isa{Incredible_Trees}).
\item To explain our abstract locales, and ensure that the assumptions are consistent, we provide
      example instantiations for them.
\end{itemize}

It does not cover the unification procedure and expects that a suitable instantiation is already
given. It also does not cover the creation and use of custom blocks, which abstract over proofs and thus correspond to lemmas in Isabelle.


\subsection*{Acknowledgements}

We would like to thank Andreas Lochbihler for helpful comments.

\bibliographystyle{amsalpha}
\bibliography{root}

\clearpage
\newcommand{\theory}[1]{\subsection{#1}\label{sec\string_#1}\input{#1.tex}}

\section{Auxiliary theories}
\label{ch\string_aux}

\theory{Entailment}
\theory{Indexed\string_FSet}
\theory{Rose\string_Tree}
\theory{Stream\string_Ext}

\clearpage
\section{Abstract formulas, rules and tasks}
\theory{Abstract\string_Formula}
\theory{Abstract\string_Rules}

\clearpage
\section{Incredible Proof Graphs}
\theory{Incredible\string_Signatures}
\theory{Incredible\string_Deduction}
\theory{Abstract\string_Rules\string_To\string_Incredible}

\clearpage
\section{Natural Deduction}
\theory{Natural\string_Deduction}

\clearpage
\section{Correctness}
\theory{Incredible\string_Correctness}

\clearpage
\section{Completeness}
\theory{Incredible\string_Trees}
\theory{Build\string_Incredible\string_Tree}
\theory{Incredible\string_Completeness}

\clearpage
\section{Instantiations}

To ensure that our locale assumption are fulfillable, we instantiate them with
small examples.

\theory{Propositional\string_Formulas}
\theory{Incredible\string_Propositional}
\theory{Incredible\string_Propositional\string_Tasks}
\theory{Predicate\string_Formulas}
\theory{Incredible\string_Predicate}
\theory{Incredible\string_Predicate\string_Tasks}

\end{document}
